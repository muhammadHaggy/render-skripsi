%-----------------------------------------------------------------------------%
\addappendix{Kode Firmware Sistem IoT}
\begin{flushright}
	Lampiran 1: Kode \f{Firmware} Sistem IoT
\end{flushright}
\label{appendix:firmware}
%-----------------------------------------------------------------------------%

Berikut adalah implementasi kode program (\f{firmware}) yang digunakan pada mikrokontroler untuk melakukan akuisisi data sensor dan penyimpanan data ke SD Card.

\begin{lstlisting}[language=C++, caption={Kode \f{Firmware} Sistem IoT - Inisiasi}, label={code:firmwareInit}]
#include <SoftwareSerial.h>
#include <SD.h>
#include <SPI.h>
#include <Wire.h>
#include <RTClib.h>
#include <MQUnifiedsensor.h>

#define BOARD "ARDUINO UNO"
#define VOLTAGE_IN 5.0
#define ADC_RESOLUTION 10

SoftwareSerial co2Serial(0, 1);
RTC_DS1307 rtc;

unsigned char hexData[9] = { 0xFF, 0x01, 0x86, 0x00, 0x00, 0x00, 0x00, 0x00, 0x79 };

const int chipSelect = 10;

const int hcPin = A0;
const int coPin = A1;

const float R0_HC = 1.13;
const float R0_CO = 1.60;

MQUnifiedsensor MQ_HC(BOARD, VOLTAGE_IN, 10, hcPin, "MQ-2");
MQUnifiedsensor MQ_CO(BOARD, VOLTAGE_IN, 10, coPin, "MQ-7");

int lastSec = -1;
\end{lstlisting}

\begin{lstlisting}[language=C++, caption={Kode \f{Firmware} Sistem IoT - Setup}, label={code:firmwareSetup}]
void setup() {
    Serial.begin(9600);
    co2Serial.begin(9600);

    Wire.begin();
    if (!rtc.begin()) {
        Serial.println("RTC failed to initialized");
        Serial.flush();
    }
    Serial.println("RTC initialized");

    if (!rtc.isrunning()) {
        Serial.println("RTC is not running. setting time to compile time");
        rtc.adjust(DateTime(F(__DATE__), F(__TIME__)));
    }

    Serial.println("Initializing SD card...");

    if (!SD.begin(chipSelect)) {
        Serial.println("SD Card initialization failed!");
    }
    Serial.println("SD card initialized.");

    File myFile = SD.open("datalog.csv", FILE_WRITE);
    if (myFile) {
        if (myFile.size() == 0) {
            Serial.println("Writing CSV Header...");
            myFile.println("Timestamp_ms,CO2_ppm,CO_ppm,HC_ppm");
        }
        myFile.close();
    } else {
        Serial.println("Error opening datalog.csv");
    }

    Serial.println("MQ Sensors Initialization");
    if (R0_HC == 0.0 || R0_CO == 0.0) {
        Serial.println("!!! WARNING: R0 values are not set. Calibrate first! Using 0 as placeholder. !!!");
        MQ_HC.setR0(1.0);
        MQ_CO.setR0(1.0);
    } else {
        MQ_HC.setR0(R0_HC);
        MQ_CO.setR0(R0_CO);
        Serial.println("R0 values loaded.");
    }

    MQ_HC.setA(574.25); MQ_HC.setB(-2.22);
    MQ_CO.setA(99.042); MQ_CO.setB(-1.518);

    MQ_HC.init();
    MQ_CO.init();
    Serial.println("Sensor MQ terinisialisasi.");
    rtc.adjust(DateTime(F(__DATE__), F(__TIME__)));
    DateTime now = rtc.now();
    lastSec = now.second();
    Serial.println("rtc initialized");
}
\end{lstlisting}

\begin{lstlisting}[language=C++, caption={Kode \f{Firmware} Sistem IoT - Loop}, label={code:firmwareLoop}]
void loop() {
    DateTime now = rtc.now();

    if(now.second() != lastSec) {
        MQ_HC.update();
        MQ_CO.update();
        float hcPPM = MQ_HC.readSensor();
        float coPPM = MQ_CO.readSensor();
        long co2PPM = readCO2Sensor();

        lastSec = now.second();

        String dataString = "";
        dataString += String(now.timestamp(DateTime::TIMESTAMP_FULL));
        dataString += ",";
        dataString += String(co2PPM);
        dataString += ",";
        dataString += String(coPPM, 2);
        dataString += ",";
        dataString += String(hcPPM, 2);

        File myFile = SD.open("datalog.csv", FILE_WRITE);

        if (myFile) {
            myFile.println(dataString);
            myFile.close();
            Serial.println(dataString);
        } else {
            Serial.println("error opening datalog.csv");
        }
    }
    delay(100);
}
\end{lstlisting}

\begin{lstlisting}[language=C++, caption={Kode \f{Firmware} Sistem IoT - Functions}, label={code:firmwareFunc}]
long readCO2Sensor() {
    co2Serial.write(hexData, 9);

    unsigned long startTime = millis();
    while (co2Serial.available() < 9) {
        if (millis() - startTime > 1000) {
            Serial.println("CO2 Sensor Timeout!");
            while(co2Serial.available()) co2Serial.read();
            return -1;
        }
        delay(5);
    }

    unsigned char response[9];
    for (int i = 0; i < 9; i++) {
        response[i] = co2Serial.read();
    }

    if (response[0] == 0xFF && response[1] == 0x86) {
        long hi = response[2];
        long lo = response[3];
        long CO2 = hi * 256 + lo;
        return CO2;
    } else {
        Serial.println("CO2 Checksum/Header Error!");
        return -2;
    }
}
\end{lstlisting}

%-----------------------------------------------------------------------------%
\addappendix{Contoh Hasil Inferensi Model (JSON Output)}
\begin{flushright}
	Lampiran 2: Contoh Hasil Inferensi Model (JSON Output)
\end{flushright}
\label{appendix:inferenceOutput}
%-----------------------------------------------------------------------------%

Berikut adalah contoh keluaran (\f{output}) JSON dari sistem inferensi yang disimpan di MinIO. Data ini menunjukkan hasil prediksi untuk tiga alternatif rute yang berbeda, lengkap dengan estimasi waktu tempuh, jarak, dan total emisi (\ce{CO2}, CO, HC).

\begin{lstlisting}[language=json, caption={Contoh Output JSON Hasil Inferensi Green Routing}, label=code:jsonInference]
{
  "run_timestamp": "2026-12-01_01-36-26",
  "vehicle": {
    "fuel_type": "diesel",
    "engine_cc": 2500
  },
  "timezone": "UTC+7",
  "departure_time_buffered_minute": 7,
  "ranking": {
    "fastest_trip_id": 1,
    "greenest_trip_id": 3
  },
  "routes": [
    {
      "trip_id": 1,
      "departure_time_now_utc+7": "2026-12-01T08:49:30",
      "google_distance_km": 41.22,
      "model_distance_km": 41.155,
      "estimated_travel_time_min": 49.2,
      "google_eta_min": 49.2,
      "model_eta_min": 49.2,
      "avg_speed_kmh": 11.68,
      "max_speed_kmh": 37.54,
      "total_CO2_g": 2577.66,
      "total_CO_g": 103.24,
      "total_HC_g": 19.66,
      "road_context_distribution": {
        "urban": 63.35,
        "arterial": 36.65
      },
      "speed_cap_distribution": {
        "0-60": 100.0,
        "60-80": 0.0,
        "80-100": 0.0
      },
      "origin": "PT Paragon Technology And Innovation Jatake 1, Jl. Raya Industri IV Blok AG No.8, Kota Tangerang, Banten 15135",
      "destination": "Green Office Park 1 BSD City, Jl. BSD Raya Bar., Sampora, Cisauk, Tangerang Regency, Banten 15345"
    },
    {
      "trip_id": 2,
      "departure_time_now_utc+7": "2026-12-01T08:49:30",
      "google_distance_km": 38.73,
      "model_distance_km": 38.672,
      "estimated_travel_time_min": 50.0,
      "google_eta_min": 50.0,
      "model_eta_min": 50.0,
      "avg_speed_kmh": 7.77,
      "max_speed_kmh": 33.51,
      "total_CO2_g": 2234.64,
      "total_CO_g": 108.98,
      "total_HC_g": 21.45,
      "road_context_distribution": {
        "urban": 63.46,
        "arterial": 36.54
      },
      "speed_cap_distribution": {
        "0-60": 100.0,
        "60-80": 0.0,
        "80-100": 0.0
      },
      "origin": "PT Paragon Technology And Innovation Jatake 1, Jl. Raya Industri IV Blok AG No.8, Kota Tangerang, Banten 15135",
      "destination": "Green Office Park 1 BSD City, Jl. BSD Raya Bar., Sampora, Cisauk, Tangerang Regency, Banten 15345"
    },
    {
      "trip_id": 3,
      "departure_time_now_utc+7": "2026-12-01T08:49:30",
      "google_distance_km": 31.39,
      "model_distance_km": 31.328,
      "estimated_travel_time_min": 51.9,
      "google_eta_min": 51.9,
      "model_eta_min": 51.9,
      "avg_speed_kmh": 10.18,
      "max_speed_kmh": 36.63,
      "total_CO2_g": 1925.06,
      "total_CO_g": 81.11,
      "total_HC_g": 15.59,
      "road_context_distribution": {
        "urban": 72.02,
        "arterial": 27.98
      },
      "speed_cap_distribution": {
        "0-60": 100.0,
        "60-80": 0.0,
        "80-100": 0.0
      },
      "origin": "PT Paragon Technology And Innovation Jatake 1, Jl. Raya Industri IV Blok AG No.8, Kota Tangerang, Banten 15135",
      "destination": "Green Office Park 1 BSD City, Jl. BSD Raya Bar., Sampora, Cisauk, Tangerang Regency, Banten 15345"
    }
  ]
}
\end{lstlisting}
\clearpage