%-------------------------%
\pagestyle{onlypage}
\chapter*{\kataPengantar}
%-----------------------------------------------------------------------------%

Puji dan syukur tim penulis panjatkan ke hadirat Tuhan Yang Maha Esa atas segala rahmat dan karunia-Nya, sehingga tim penulis dapat menyelesaikan penyusunan tugas akhir ini dengan baik dan tepat waktu. Dalam proses penyusunan laporan tugas akhir ini, tim penulis memperoleh banyak bantuan, dukungan, dan bimbingan dari berbagai pihak. Oleh karena itu, dengan segala kerendahan hati, tim penulis menyampaikan terima kasih yang sebesar-besarnya kepada:

\begin{itemize}[topsep=0pt,itemsep=0ex,partopsep=1ex,parsep=1ex]
    \item Keluarga Penulis 1, yaitu Mamah Lia, Abah Hasyim, Sheirin, dan Sophia yang selalu mendoakan dan memberikan dukungan kepada Penulis 1 selama perkuliahan hingga akhirnya menyelesaikan penelitian ini.
    \item Keluarga Penulis 2, yaitu Ayah Dinan, Ibun Rustiyah, dan Nadiv yang senantiasa memberikan dukungan kepada penulis 2 dalam menempuh dan menyelesaikan pendidikan di perguruan tinggi.
    \item Keluarga Penulis 3, yaitu Ayah Eko, Ibu Ussa, Bude Yun, dan Uti yang senantiasa mendorong, memberi semangat dan dukungan, serta membantu penulis 3 dalam menjalani dan menyelesaikan pendidikan perguruan tinggi.
    \item Dr. Dr. Ari Wibisono, S.Kom., M.Kom. dan Syifa Nurhayati, S.Kom, M.Kom. selaku dosen pembimbing yang telah memberikan bimbingan, masukan, serta motivasi selama proses penelitian dan penulisan laporan tugas akhir ini.
    \item Teman-teman ``Homies in C.....'', Vei, Arkun, Gib, Otsu, Dep, Jems, Josh, Alek, Napis yang telah mewarnai perkuliahan penulis 1 dan memberi kenangan yang baik.
    \item SASsy yang setia menemani dan menjadi tempat berbagi keluh kesah bagi penulis 1 selama masa perkuliahan dan pengerjaan skripsi.
    \item Chandra, Mas Rachmat, dan Nabiel yang menemani dan mewarnai kegiatan magang penulis 1 di Bank Indonesia dan menjadi tempat keluh kesah dan curhat galau selama masa perkuliahan.
    \item Obadah Yusuf, Mas Adib, Bli Wayan, Fancy, Titi yang sudah membantu beradaptasi, berjuang \f{survive} bersama, dan mewarnai kehidupan penulis 1 selama merantau di Korea Selatan.
    \item Abid, Bang Aldy, Bang Sergio dan Ardhito yang telah mewarnai kehidupan perkuliahan penulis 1 di dunia Tim Robotika UI (Reignblaze).
    \item Kak Firda dan Chiesa yang sudah menjadi tempat penulis 2 mencurahkan segenap isi pikiran dan keluh-kesah selama ini.
    \item Teman-teman ``Excalibur Wielders'' termasuk Abdul Rafi, Muhammad Hadziq Razin, Muhammad Falensi Azmi, Valencius Apriady, Jason Irvine Mahendra Putra, dan Muhammad Hanif yang sudah senantiasa berdiskusi, berdebat, dan saling merangkul selama perkuliahan penulis 2 berlangsung.
    \item Naufal, Sam, Fredo, Kenichi, Tommy, Hilmy, serta mentor dan rekan Ristek Gamedev yang berjasa dalam proses perkembangan pola pikir dan pengetahuan penulis 2 di dunia \f{game development}.
    \item Robby, Dhika, dan rekan-rekan Sporteev yang sudah mendukung penuh penulis 2 dalam penelitian kali ini dan memberikan memori serta pengalaman profesional pertama yang baik di luar kegiatan kampus.
    \item Segenap pengurus Departemen Olahraga BEM Fasilkom UI Tahun 2023 dan 2024, pengurus UKOR Renang tahun 2023 dan 2024, serta kordinator kontingen renang Fasilkom UI dalam rangka mempersiapkan serta menemani penulis 2 dalam rangka mengikuti pertandingan OLIM UI cabang olahraga renang pada tahun 2023 dan 2024 secara berturut-turut. 
    \item Rekan seperjuangan divisi Indie Game Ignite COMPFEST 14 dan COMPFEST 15 yang sudah berupaya untuk menghidupkan komunitas \f{game developer} bersama di lingkungan Universitas Indonesia, terkhususnya Fasilkom.
    \item Alicia, Nataya, Adib, Athallah, Alpha dan Ardian yang sudah menjadi rekan berdiskusi dalam perancangan konsep dan implementasi desain sistem IoT yang kemudian penulis 2 gunakan dalam penelitian kali ini melalui bidang pengetahuannya masing-masing.
    \item Rekan-rekan dari ``Original Mit'', ``Dark Glory Sloer'', dan ``LAPENDOS'' yang sudah menjadi teman bermain penulis 2 dan menjadi tempat bagi penulis 2 untuk rehat dari seluruh hiruk-pikuk kehidupan akademis.
    \item Pascal, Nabil, dan Kohan yang menjadi partner dan kunci bagi penulis 2 untuk bisa bertahan di semester ini.
    \item Rekan-rekan Sinematografi UI dan Computer Science Sineas yang sudah menjadi media pembelajaran yang baik melalui baik dan buruknya kejadian yang dialami dan dihadapi bagi penulis 2.
    \item Segenap rekan dari ``Warung Es Kelapa Muda Bu Aas'' yang tidak bisa disebutkan satu persatu.
    \item Naznien, Joan, dan Rafinal yang telah menjadi sistem pendukung terbaik, memberikan motivasi tanpa henti, serta menjadi tempat berbagi keluh kesah bagi penulis 3 selama masa perkuliahan.
    \item Wedens, Carol, Shamira, dan Refiany, teman kerja selama magang di Traveloka yang telah mendukung dan menyemangati penulis 3.
    \item Teman-teman ``Sirkel Tennis'', yaitu Rafinal, Naznien, Dimitri, Fikri, Akmal, Adit, dan Raihan yang telah menemani penulis 3 menjalani kehidupan kampus dan memberi memori baik.
    \item Jasmine, Asnat, Kaylee, Aliya, Alanna, Mayfa, Jihan, Ayuna, Farah, Galen, Kezia Lasma, Kezia Natalia, Afsar, Najmi, Martin, Daffa, Upi yang telah menemani penulis 3 menjalani kehidupan kampus dan memberi memori baik.
    \item Bibi, Rara, Mikella, Amel, Anggi, Naura, Cia, Adra, Rifqi, dan Aaron, yang selalu hadir dan memberikan dukungan penuh kepada penulis 3, baik dalam bentuk semangat maupun waktu yang sangat berarti selama menjalani perkuliahan.
\end{itemize}

Tim penulis menyadari bahwa tugas akhir ini masih memiliki keterbatasan. Oleh karena itu, penulis sangat terbuka terhadap saran dan kritik yang membangun guna penyempurnaan karya ini di masa mendatang. Besar harapan tim penulis agar tugas akhir ini dapat memberikan manfaat, baik secara akademik maupun praktis, terutama dalam upaya peningkatan kualitas layanan di lingkungan Fakultas Ilmu Komputer Universitas Indonesia.

\vspace*{0.1cm}
\begin{flushright}
Depok, 11 Desember 2025\\[0.1cm]
\vspace*{1.5cm}
Tim Penulis
\end{flushright}