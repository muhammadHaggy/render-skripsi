%
% @author  Andre Tampubolon, Andreas Febrian
% @version 2.2.0
% @edit by Muhammad Aulia Adil Murtito, Ichlasul Affan
%

\chapter*{\uppercase{Halaman Pernyataan Persetujuan Publikasi\\Tugas Akhir untuk Kepentingan Akademis}}
\vspace*{-1cm}
\par\noindent\rule{\textwidth}{3pt}
\vspace*{1cm}
\noindent
Sebagai sivitas akademik Universitas Indonesia,~\ifx\blank\npmDua{saya}\else{kami}\fi~yang bertanda
tangan di bawah ini:

\vspace*{0.2cm}

\pagestyle{onlypage}

\begin{tabular}{p{4.2cm} l p{6.5cm}}
	\ifx\blank\npmDua
	Nama&: & \penulisSatu \\
	NPM&: & \npmSatu \\
	Program Studi&: & \programSatu\\
	\else
	\bo{Penulis 1}\\
	Nama&: & \penulisSatu \\
	NPM&: & \npmSatu \\
	Program Studi&: & \programSatu \vspace*{0.1cm}\\
	\bo{Penulis 2}\\
	Nama&: & \penulisDua \\
	NPM&: & \npmDua \\
	Program Studi&: & \programDua \vspace*{0.1cm}\\
	\fi
	\ifx\blank\npmTiga\else
	\bo{Penulis 3}\\
	Nama&: & \penulisTiga \\
	NPM&: & \npmTiga \\
	Program Studi&: & \programTiga \vspace*{0.1cm}\\
	\fi
	Jenis Karya & : & \type \\
\end{tabular}

\vspace*{0.2cm}

\noindent demi pengembangan ilmu pengetahuan, menyetujui untuk memberikan
kepada Universitas Indonesia \bo{Hak Bebas Royalti Noneksklusif
(\textit{Non-exclusive Royalty Free Right})} atas karya ilmiah~\ifx\blank\npmDua{saya}\else{kami}\fi~yang berjudul:
\begin{center}
	\judul
\end{center}
beserta perangkat yang ada (jika diperlukan). Dengan Hak Bebas Royalti
Noneksklusif ini Universitas Indonesia berhak menyimpan,
mengalihmedia/formatkan, mengelola dalam bentuk pangkalan data
(\textit{database}), merawat, dan memublikasikan tugas akhir~\ifx\blank\npmDua{saya}\else{kami}\fi~selama
tetap mencantumkan nama~\ifx\blank\npmDua{saya}\else{kami}\fi~sebagai penulis/pencipta dan sebagai
pemilik Hak Cipta. \\

\noindent Demikian pernyataan ini~\ifx\blank\npmDua{saya}\else{kami}\fi~buat dengan sebenarnya.

\ifx\blank\npmDua\else\clearpage\fi

\begin{center}
	\vspace*{0.8cm}
	\begin{tabular}{lll}
		Dibuat di&: & Depok \\
		Pada tanggal&: & \tanggalSiapSidang \\
	\end{tabular}\\

	\vspace*{0.2cm}
	Yang menyatakan \\
	\ifx\blank\npmDua
        % --- KASUS 1 PENULIS ---
		\vspace*{0.5cm}
		\includegraphics[height=1.5cm]{assets/pics/ttd_1.png}\\
		(\penulisSatu)
	\else
        % --- KASUS 2 PENULIS ---
		\begin{multicols}{2}
			Penulis 1:\\
			\vspace*{0.5cm}
			\includegraphics[height=1.5cm]{assets/pics/ttd_1.png}\\
			(\penulisSatu)\\

			Penulis 2:\\
			\vspace*{0.5cm}
			\includegraphics[height=1.5cm]{assets/pics/ttd_2.png}\\
			(\penulisDua)\\
		\end{multicols}
	\fi
	\ifx\blank\npmTiga\else
        % --- KASUS 3 PENULIS ---
		\vspace*{0.2cm}
		Penulis 3:\\
		\vspace*{0.5cm}
		\includegraphics[height=1.5cm]{assets/pics/ttd_3.png}\\
		(\penulisTiga)
	\fi
\end{center}

\newpage