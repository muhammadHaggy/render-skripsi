%
% @author  Andreas Febrian
% @version 2.2.1
% @edit by Ichlasul Affan
%
% Mendaftar seluruh istilah yang mungkin akan perlu dijadikan
% italic atau bold pada setiap kemunculannya dalam dokumen.
%

% Istilah/alias yang tidak perlu dimasukkan ke dalam Glossary/Daftar Istiilah
\var{\license}{\f{Creative Common License 1.0 Generic}}
\var{\bslash}{$\setminus$}

\makeglossaries

% --- AKRONIM ---
\newacronym{vrptw}{VRPTW}{\f{Vehicle Routing Problem with Time Windows}}
\newacronym{iot}{IoT}{\f{Internet of Things}}
\newacronym{mlops}{MLOps}{\f{Machine Learning Operations}}
\newacronym{vsp}{VSP}{\f{Vehicle Specific Power}}
\newacronym{mcm}{MCM}{\f{Markov Chain Model}}
\newacronym{lidar}{LiDAR}{\f{Light Detection and Ranging}}
\newacronym{rmse}{RMSE}{\f{Root Mean Square Error}}
\newacronym{mape}{MAPE}{\f{Mean Absolute Percentage Error}}
\newacronym{gnss}{GNSS}{\f{Global Navigation Satellite System}}
\newacronym{bpp}{BPP}{\f{Bin Packing Problem}}

% --- ISTILAH ---
\newglossaryentry{green_logistics}
{
	name={Green Logistics},
	description={Praktik logistik yang bertujuan meminimalisir dampak ekologis dari kegiatan logistik, mencakup transportasi, pergudangan, dan distribusi}
}

\newglossaryentry{movestar}
{
	name={MOVESTAR},
	description={Model estimasi emisi kendaraan berbasis daya spesifik kendaraan (VSP) yang dikembangkan untuk menangkap dinamika emisi skala mikro}
}

\newglossaryentry{legacy_system}
{
	name={Legacy System},
	description={Sistem perangkat lunak atau perangkat keras lama yang masih digunakan dalam organisasi, meskipun teknologi yang lebih baru sudah tersedia}
}

\newglossaryentry{ground_truth}
{
	name={Ground Truth},
	description={Informasi yang diperoleh melalui pengamatan langsung (pengukuran empiris) yang digunakan sebagai acuan kebenaran untuk memvalidasi model}
}